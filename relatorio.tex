\documentclass[a4paper,11pt]{article}
\usepackage[top=3cm,right=2.5cm,bottom=2.5cm,left=3cm]{geometry}
\usepackage[brazil]{babel} 
\usepackage[utf8]{inputenc}
\usepackage{indentfirst} 
\usepackage{graphicx}
\usepackage{here}


\begin{document}

\title{{\it n-puzzle} \\
  Primeiro trabalho da disciplina de \\
  Inteligência artificial}

\author { André Parreira Malaguini\\
  Diego Giovane Pasqualin\\
  Leonardo Fedalto\\
  Ulysses Bonfim}

\date{\today}
\maketitle
  

\section{Problema proposto}

Seja uma matriz quadrada de tamanho {\it n} preenchida aletoriamente por 
{\it $n^2 - 1$} 
números sendo que, uma destas posições contém um espaço em
``branco''. O objetivo do jogo é, trocando o espaço em branco apenas pelas peças
que são não-diagonalmente tangíveis, levar a matriz à uma configuração tal que
os números dispostos estão ordenados, primeiramente pela linha e depois por
coluna, e o último é o caracter nulo.

% Por um Here nas tabelas
\begin{table}[H]
  \begin{center}

    \begin{tabular}{c c c c}
      \begin{tabular}{|c|c|c|}
        \hline
        9 &5 &1 \\ \hline
        4 &3 &7 \\ \hline
          &2 &8 \\ \hline
      \end{tabular}
      & & &

      \begin{tabular}{|c|c|c|}
        \hline
        1 &2 &3 \\ \hline
        4 &5 &6 \\ \hline
        7 &8 &  \\ \hline
      \end{tabular}
    \end{tabular}

  \end{center}

  \caption{Instância de um problema onde {\it $n = 3$}, e a respectiva solução 
    para os tabuleiros de tamanho {\it n}.}

\end{table}


\section{Resolução}
Para solucionar o problema computacionalmente, devemos representá-lo de forma
que possamos manipular os dados e encontrar uma solução utilizando um raciocínio
que simule o comportamento humano. Uma maneira para representar os dados e
avaliar as possíveis jogadas, é utilizar a estrutura de árvore. Cada nodo da árvore contém uma configuração de tabuleiro, que nada mais é que uma lista de listas.
Deste modo, partindo da posição inicial, podemos expandir todas as jogadas possíveis e fazer
isto recursivamente até encontrar a solução ({\it busca em largura}), ou
expandimos apenas uma das jogadas e continuamos desta forma até que o ramo
escolhido nos leve a uma das soluções ou, caso esse ramo seja falho, voltamos e
escolhemos outro nó para ser expandido ({\it busca em profundidade}).

Em nenhum dos dois métodos acima levamos em consideração o jogo em questão, logo
a busca que fazemos na árvore não é inteligente. Sabendo de antemão as regras do
jogo e alguma maneira de encontrar a vitória mais rápido, podemos não
expandir os
nós de maneira indiscriminada, o que pode conduzir a uma vitória mais rápida. A
estimativa do custo do caminho entre um nó e o objetivo é chamada de {\it função
  heurística}.

\section{Percorrendo a árvore}
Para percorrer a árvore de possíveis estados do jogo, escolhemos o método
\textit{A*} que
leva em consideração, não somente a função heurística mas também a altura que
estamos na árvore. Dessa maneira não nos perderemos em um caminho onde todas as
jogadas tem um peso baixo porém precisaremos de muitas, ou infinitas, para chegar
a vitória.

Neste método, dado um nó inicial, os seus filhos são avaliados seguindo uma
função peso {\it f(x)}, $ f(x) = g(x) + h(x) $, onde $ g(x) $ é a altura do nó
{\it x}, e $h(x)$ seu valor heurístico.


\section{Implementação da solução}
Partindo de uma configuração inicial qualquer, avaliamos os pesos de todas as
jogadas possíveis, ou seja, os filhos do nodo inicial, usando alguma função heurística, e inserimos em uma lista ordenada pelos
pesos. A partir dai, expandimos o nodo que está em primeiro na lista ordenada.
As jogadas subsequentes seguem a mesma rotina, até encontrar o tabuleiro de peso 0, que é a solução.

%Para evitar os loops de jogadas repetidas guardamos todos os tabuleiros gerados
%em uma outra lista, assim quando um novo filho é gerado só o inserimos na lista
%de filhos se o movimento não levar a uma configuração já alcançada.

\subsection{Funções heurísticas e comparações}
Como já citado acima, o papel da função heurística é fazer com que a tomada de
decisão na expansão dos nós nos leve a uma vitória, de uma maneira mais inteligente
do que apenas expandir a árvore de forma indiscriminada.

A seguir, discutiremos as heurísticas usadas e faremos uma comparação entre elas

\subsubsection {Distância Manhattan}
A primeira tentativa de implementação foi usando como heurística a distância Manhattan.
Antes de escolher um nodo para expandir (ou seja, uma jogada), primeiro calculamos a distância entre a
posição atual e a posição onde a peça deveria estar, para todas as peças do tabuleiro. Somamos então todas as distâncias com a altura
na árvore em que está o tabuleiro. Fazemos isso para todos os nodos filhos do nodo inicial.
Então, colocamos o nodo da árvore que contém o tabuleiro na lista ordenada de nodos, para então sim escolhermos a próxima jogada.
%por alguma razão, essa heurística não está resolvendo problemas de 4x4. (nao sei se coloca isso)

\subsubsection {Heurística da "Cebola"}
A fim de fazer o menor número de jogadas possíveis, pensamos em uma implementação que resolvesse primeiro as partes mais externas do tabuleiro.
Dessa forma, essas peças mais externas não precisariam ser mais movidas. Para o programa atingir um comportamento que resolvesse dessa forma,
adicionamos um elemento à heurística anterior, uma matriz multiplicadora com a seguinte configuração:

\begin{table}[H]
  \begin{center}

    \begin{tabular}{c c c c c c c c}
      \begin{tabular}{|c|c|c|c|}
        \hline
        {\it n-1} &{\it n-1} &{\it n-1} &{\it n-1} \\ \hline
        {\it n-1} &{\it n-2} &{\it n-2} &{\it n-2} \\ \hline
        {\it n-1} &{\it n-2} & ...  & ...  \\ \hline
        {\it n-1} &{\it n-2} & ...  & 0 \\ \hline
      \end{tabular}

      & &
      \begin{tabular}{|c|c|c|c|}
        \hline
        3 &3 &3 &3 \\ \hline
        3 &2 &2 &2 \\ \hline
        3 &2 &1 &1 \\ \hline
        3 &2 &1 &0 \\ \hline
      \end{tabular}

      & &

      \begin{tabular}{|c|c|c|c|}
        \hline
        7 &1 &11&10\\ \hline
        15&13&  &6 \\ \hline
        12&8 &9 &2 \\ \hline
        14&4 &3 &5 \\ \hline
      \end{tabular}
    \end{tabular}

  \end{center}

  \caption{Matriz multiplicadora para {\it n} qualquer, uma instância dela para
    {\it $n=4$} e um exemplo de tabuleiro. Assim, para obtermos o valor
    heurístico do número 12, por exemplo, calculamos a distância até sua posição
    correta, valor 3, e multiplicamos pelo valor contido na 12ª posição (linha 
    3, coluna 3) da matriz multiplicadora. O resultado neste
    exemplo é $h(12) = 3 * 1 = 3$.}

\end{table}

Agora, para cada peça do tabuleiro, calcula-se a distância manhattan
multiplicada pelo valor correspondente na matriz multiplicadora.
Soma-se então todos esses valores com a altura do nodo na árvore.

\subsubsection {Heurística da "cebola" com soma}
%nao sei o motivo de usar essa heuristica. foi a ideia inicial?
Essa heurística funciona da mesma forma que a Herística citada anteriormente, com a diferença que em vez de multiplicar a distância manhattan com o valor na matriz multiplicadora, faz-se a soma.

\subsubsection {Heurística Layers}
Tentando simular o pensamento normal do ser humano, foi feita a implementação com esta heurística.
A idéia é dar prioridade a primeira linha, depois a segunda, e assim por diante.
O algoritmo é o mesmo usada na Heurística da "cebola", com a diferença de que a matriz multiplicadora usa a seguinte configuração:

\begin{table}[H]
  \begin{center}

    \begin{tabular}{c c c c c c c c}
      \begin{tabular}{|c|c|c|c|}
        \hline
        {\it n} &{\it n} &{\it n} &{\it n} \\ \hline
        {\it n-1} &{\it n-1} &{\it n-1} &{\it n-1} \\ \hline
        {\it n-2} &{\it n-2} & ...  & ...  \\ \hline
        {\it 1} &{\it 1} & ...  & 0 \\ \hline
      \end{tabular}

      & &
      \begin{tabular}{|c|c|c|c|}
        \hline
        4 &4 &4 &4 \\ \hline
        3 &3 &3 &3 \\ \hline
        2 &2 &2 &2 \\ \hline
        1 &1 &1 &0 \\ \hline
      \end{tabular}

      & &

      \begin{tabular}{|c|c|c|c|}
        \hline
        7 &1 &11&10\\ \hline
        15&13&  &6 \\ \hline
        12&8 &9 &2 \\ \hline
        14&4 &3 &5 \\ \hline
      \end{tabular}
    \end{tabular}

  \end{center}

  \caption{Matriz multiplicadora para {\it n} qualquer, uma instância dela para
    {\it $n=4$} e um exemplo de tabuleiro. Assim, para obtermos o valor
    heurístico do número 12, por exemplo, calculamos a distância até sua posição
    correta, valor 3, e multiplicamos pelo valor contido na 12ª posição (linha 
    3, coluna 3) da matriz multiplicadora. O resultado neste
    exemplo é $h(12) = 3 * 2 = 6$.}

\end{table}
\end{document}
